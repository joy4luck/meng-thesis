%% This is an example first chapter.  You should put chapter/appendix that you
%% write into a separate file, and add a line \include{yourfilename} to
%% main.tex, where `yourfilename.tex' is the name of the chapter/appendix file.
%% You can process specific files by typing their names in at the 
%% \files=
%% prompt when you run the file main.tex through LaTeX.
\chapter{Introduction}
  Social support is an important resource during times of transition and stress.
  Whether it's friends, collegues, families, or anyone else, the people in our lives
  are what help get us through difficult transitions.
  \cite{mikal13}
  The impact of this support has manifested itself in a broad range of studies.
  Rains and Young found that supportive communication can help manage uncertainty and
  help one feel in control of one's life \cite{rains09},
  and Shor et all even discovered that support from
  family is correlated with lower mortality. \cite{shor13}

  However, as the complexity of personal networks suggests, the resources that people pull on
  are very specific,
  and their appropriateness is a complex subject.
  It is thus important to recognize that not all support is the same.
  Lehman et al studied in 1986 the ways that attempts at support that were helpful and unhelpful.
  \cite{lehman86}
  When Vachon and Stylianos explored the "goodness of fit" between offered support and the actual needs of
  the bereaved, they discovered that it varies wildly, and is sensitive to variables such as impact of loss
  and presence of other stressors.
  \cite{vachon88}
  
  Further, stressful transition periods can be accompanied by changes in the social structure
  that the support depends upon.
  The situation of illness or bereavement obviously includes, necessarily,
  the loss of an individual that would otherwise be a source of support,
  but many other transitions can also cause rearrangement of personal networks.
  Consequently, during transitions is often when these much needed
  support mechanisms can weaken or fail.
  \cite{mikal13}

  In this thesis, I am interested in exploring
  the ways that technology actually helps individuals going
  through stressful transitions.

  This work fits into the broader universe of
  recent Human Computer Interaction research
  done to address the need to mediate social support
  and other forms of intimacy through technology.
  \cite{hassenzhal12}
  Technology can help to mediate the appropriate type of resilient support
  that individuals may need.
  Promisingly, there has been significant evidence that
  mobile and web based social interactions can have positive effects similar
  to more traditional social interactions.
  \cite{??}

\section{Overview}
  The work of this thesis is split into roughly three different areas.
  \begin{enumerate}
  \item \textbf{Interviews and Questionnaires} were conducted to better understand
    university students' existing interaction with intimate socials
    such as family or friends while going through a stressful period in life.
  \item \textbf{Field Tests} of a technological probe, the InMind Android application,
    over 3 weeks
    explore the concept of a life issue tracker that is shared with friends.
    Data that was logged was used to answer questions about privacy choices that users make.
  \item \textbf{Feedback and Reflection} on the usage of the InMind application,
    as well as the evolving nature of the participant' situations will
    be used to motivate future design.
  \end{enumerate}

\section{Project Vision and History}
\label{sec:vision}
  InMind began as an application designed to help bereaved individuals deal with
  some of their specific needs, specifically story, affect, and availability sharing.

  My initial prototype, InMourning, targeted three needs that became apparent
  from needfinding studies conducted by Massimi and
  Baecker \cite{mm11a, mm10, mm13},

  \begin{enumerate}
  \item \textbf{Affect Sharing} - The well being of bereaved individuals is often a subject
    of inquiry, and sharing this information is a common task.
  \item \textbf{Availability Sharing} -
    Caring friends and family often want to be present
    and available for the bereaved, however,
    the bereaved generally want to control the nature and timing of communication between
    them and their family, friends, and other supporters.
    The bereaved are sensitive to what kind
    of interaction (texts, calls, face-to-face visits) and when they occur (exactly
    what hour of what day, or frequency). The influx of media and ways to
    communicate actually make it consistently overwhelming for the bereaved,
    particularly added to the already stressful interactions that necessarily
    follow a loved one's passing.
    Thus, the availability of the bereaved, how able and willing they are
    to accommodate guests, is important to communicate.
  \item \textbf{Story Sharing} - Bereaved individuals feel a need to share stories related to
    or inspired by their relationship with the departed.
    From reflecting on a trigger, pattern, or thought that carries unusual
    significance for the mourner to full blown story telling, bereaved individuals
    consistently want to share aspects of their experience with caring friends and
    family, many of whom are likely mourning the same loss.
  \end{enumerate}

  These three needs were collected because they addressed the common theme of improving
  relationships with caring friends and family during the time of difficulty.

  In the context of these needs, I developed InMourning to work in an HCI study,
  in the theme of existing bereavement technologies,
  hoping to better understand how people can use technology to meet their needs.
  The initial prototypes of InMourning are attached to this in Appendix C.

  However, the more we explored the needs of bereaved individuals,
  and how to best reach out to them, the more we realized that these needs exist in some form
  for all people who are dealing with the difficult periods of life.
  Problems such as chronic illness, physical or psychological,
  can make the wellness, availability, and story of individuals relevant to those
  close to them, as well.
