\chapter{Discussion}
\section{Existing Support Needs}
	In this work, we took a small sample of MIT affiliates and interviewed
	them to better understand their support needs and strategies for
	obtaining them.
	We discovered that the degree of need varies among individuals,
	and although some prefer to have close connections,
	with whom they connect and share many personal feelings,
	others do not.

\section{Privacy Concerns of Individuals}
	The privacy concerns of the individuals in the study varied as well.
	Although most were very sensitive to who is able to read what they write,
	surprisingly, a good number are almost completely neutral as to the
	degree of exposure their writing receives.
	Some of the participants actually expressed a desire to place
	the status material in a public location on the web,
	where anyone with a link could get access to it.

\section{Value of Topic Based Communication}
	The biggest contribution of this work was to bring topic-based communication,
	in the style of forums, to a personal network.
	The degree of interest in this structure varied among the participants.
	The two heaviest users expressed value in organizing things for themselves
	in a topic based way, and using the icons to communicate for themselves,
	but only one user expressed an affinity towards symbology with the people
	within the group.
	It's very likely that symbology for topics within normal life only
	connects with a subset of the population.

\section{Designing Within an Existing Technological Space}
	Further, by the many feature requests that the participants made throughout
	the study, I conclude that modern users expect a certain
	large bucket of features that accompany any communications application.
	Standards set by Facebook, Google Hangouts, and Vine lead users to expect
	a certain feature set that includes rich content such as
	a large set of emoticons and the ability to share pictures, video, and sound.
  
\section{Personal Reflections and Future Directions}
  \subsection{Weaknesses of Observation Within a Study}
		After conducting this user study, I reflected upon the difficulty of
		conducting a study on the strong-tie relationships between
		loved ones, particularly in a university setting.

		Trying to recruit participants' close ties is difficult.
		To actually observe the real connections that people are invested in,
		being involved with existing communities may be the best way
		to understand their communities.

    
