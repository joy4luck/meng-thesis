\chapter{Results}
\section{Users}
  In total, 15 people participated in the user study.
  They were recruited from MIT students,
  via dorm and living group mailing lists.

  All of the 15 people participated in the opening and closing interviews.
  However, only 8 of them participated in the Android portion of the study.
  The reasons for this will be discussed in the Section \ref{sec:Android}.

  The participants were compensated for their degree of participation,
  and the compensation structure is outlined in a table in Appendix A.

\section{Opening Questionnaire Results}
  The results from the opening questionnaire help to characterize the group's
  general thoughts on the subject of stress and support.
  
  Part of the survey included questions, scored on a Likert Scale,
  addressing to what degree they reach out to friends or family when troubled,
  and to what degree they feel supported by friends and family.
  The results are shown in Figure \ref{fig:likert}.

    \begin{figure}
    \centering
    \includegraphics[width=0.6\textwidth]{likert.png}
    \caption{
      Results of Likert scale styled questions on the opening questionnaire.
      TrFa and TrFa asked whether participants want to talk to
      Friends or Family when Troubled.
      SuppFa and SuppFa asked whether participants felt supported by
      Friends or Family.
      TrSupp asked if they reached out, in general, when troubled.
    }
    \label{fig:likert}
    \end{figure}

  Also in the opening questionnaire, we administered a Perceived Stress Scale.
  The results, shown in a histogram in Figure \ref{fig:perceived_stress},
  show that in general, the users in this study had relatively low
  self-reported stress.

    \begin{figure}
    \centering
    \includegraphics[width=0.6\textwidth]{perceived_stress.png}
    \caption{
      Distribution of Perceived Stress scores amongst the 15 participants.
    }
    \label{fig:perceived_stress}
    \end{figure}

\section{Opening Interview: Script Based Results}
  The opening interview, as mentioned in Section \ref{sec:opening_int},
  followed a script, so I'll briefly outline some of the
  prominent results from the 5 sections.

  \subsection{Support-Interest}
  On discussing the support network of the interviewees,
  they tended to have a certain group of people that they turned to for
  general day-to-day support.
  These people could be family, or family-like people, as well as friends,
  but most times, participants had a strong preference towards one or the other.
  In most cases, the interviewee chose to discuss serious subjects,
  about things that were really on their minds,
  with only friends or only family,
  and chose to discuss only mundane, less serious topics with the other.

  On general degree of need and interest in reaching out,
  the interviewees covered a pretty broad range,
  and I separated them into what I considered low support-interest
  and high support-interest groups,
  support-interest defined as having a consistent desire and efforts to
  connect on an emotional and empathetic level with loved ones.
  To clarify, support-interest is a term I'll use when describing my participants,
  and is not a diagnostic term and definitely is not known to be
  correlated with other health and wellness outcomes.
  As far as I know, it is an indicator that becomes generally obvious
  on interview with the participants,
  and only represents the current situation of the interviewee, in snapshot.

  On the low support-interest end, four interviewees seemed especially independent,
  expressing a desire for and general pattern of dealing with things
  themselves, without much external input,
  and preference for keeping their concerns to themselves.
  They were by no means not social,
  all four reported having plenty of friends that they interacted with regularly,
  but they chose to keep them at a greater distance when it came to things
  they were particularly sensitive about.
  This usually became obvious when 
  On asking about how they discuss serious subjects with those closest to them,
  they tend to express a lack of interest.
  Quotes associated with these people included:
  \begin{itemize}
  \item ``She's such a drama queen, and I don't want her to worry."
  - Participant 1496, about her mother.
  \item ``Everyone has the same issues, why talk about it?"
  - Participant 2928, about her friends.
  \item ``We only talk about mundane things...
  things not directly related to life."
  - Participant 1010, about both her parents.
  \end{itemize}

  On the high support-interest end, there was a lot of variability.
  High support-interest people included my married interviewees, who notably,
  married within the last 5 years or so,
  and expressed being very open with their spouses, sharing everything.
  They also included a participant very active in her church community,
  and other individuals close with their parents or significant others.
  These participants expressed spending a lot of time invested in
  sharing feelings and struggles.
  Quotes associated with them included:
  \begin{itemize}
  \item ``We discuss the psychological aspect of things...
  the source and causes of emotions."
  - Participant 8594, about her mentor figure.
  \item  ``I know how she's feeling with about a 15 minute resolution."
  - Participant 3792, about his girlfriend.
  \item ``He knows me when I get negative, and jumps on it."
  - Participant 4416, about her boyfriend.
  \item ``I have accountability partners, and we keep each other honest."
  - Participant 5397, about her certain members of her church community.
  \end{itemize}

  \subsection{Serious Situation}
  Although the call for participant recruitment called for participants
  dealing with some serious situation in their lives,
  whether mourning or illness or other unusual difficulty of transition,
  the severity of the participants' situations varied.

  On the lighter end of the spectrum included six participants who were
  dealing with regular academic and career pressures.
  On the heavier end,
  three participants were dealing with very serious chronic illnesses,
  another was bereaved, and other were battling some daunting transitions.

  \subsection{Technology}
  On technology,
  my participants tended to be technically savvy.
  This is likely due to the age range of the participants,
  the youngest of which was 18, and the oldest 30.
  They expressed patterns that were consistent with previous research,
  \ref{???}
  in that they used a complex range of technologies to connect
  with friends and family near and at distance,
  for a range of complicated reasons.
  More convenient forms of communication,
  such as text or other forms of instance message,
  were used frequently for casual things and keeping in touch,
  and other modes of communication, such as video, phone, or in person meetings,
  were used for more serious topics.
  The exception to this was email,
  which despite being digital and text based, 
  afforded a level of thoughtfulness that participants
  felt were especially valuable.
  One participant found emails the best medium for reconnecting
  with distant friends on the nature and current condition of her
  chronic illness,
  and another participant

\section{Opening Interview: Other Trends}
  The trends addressed by the script of the opening interview touched on
  some important topics, but the most interesting results
  were other patterns and phenomena
  that were not anticipated by the script.

  \subsection{Direction of Awareness}
  In many cases, there was a clear directionality to the type of awareness
  information shared.
  One recurrent example was with parents,
  when the daughter or son tends to report a lot,
  and the parents relatively little in return.
  Other relationships, such as between siblings or friends,
  can also be very directional,
  with one party in the habit of disclosing more to the other.

  \subsection{Extreme Information Withholding}
  Another phenomenon that became apparent with interviews is that
  many, even normally trusting, close relationships,
  can incorporate extreme instances of information withholding.
  One relatively whimsical example was brought up
  when the topic of awareness and updates from family was brought up.
  \textit{
    My dog died and they didn't tell me.
    I came home and asked "Where's the dog?"
    It's like they expected me to not notice.
  } - Participant 2928

  It's important to note that the story is one instance of a general
  trend of not sharing bad news.
  This can be as simple as preferring to look on the bright side,
  "My Facebook page used to be full of sad things.
  I decided not to do that anymore." Participant 3792
  or "I'd rather talk about happy things. Not get them down."
  - Participant 9451
  
  This can also be taken to the extreme.
  One specific example from my participants involved
  completely not telling family members about a serious issue,
  while maintaining communications otherwise.

  \subsection{Changes in Support Structure}
  Lastly, closely related to the extreme information withholding,
  there are many situations in our lives that result in
  drastic changes in the nature of our support structure.
  The drastic changes that I observed in interviews
  tended to involve a falling out of some sort.
  These could be between family members,
  significant others, or even friends.
  Another notable cause for a drastic change in support structure,
  if less common and much more tragic: death.
  When the person who passed away had a pivotal role in the support
  structure of another, it can be even harder to cope with
  the difficult period of mourning.

\section{Android}
\label{sec:Android}

\section{Data and Questions to Answer}

  Using server data:
  \begin{enumerate}
    \item When are topics made?
    \item How many topics per week?
    \item How many messages per week?
    \item Who receives topics?
    \item How long does a topic live?
  \end{enumerate}

  Using Google Analytics App data:
  \begin{enumerate}
    \item How much time is spent on the app?
    \item How much of that is with individual plants/writing notes?
    \item What time of day is the app accessed?
    \item For browsing vs status updating?
  \end{enumerate}
