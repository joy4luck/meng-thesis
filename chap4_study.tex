\chapter{User Study}
To evaluate the relevance of InMind as well as better understand
the social support needs of individuals,
I conducted a user study which was the bulk of the work for this thesis.

In this section, I will broadly outline the specific goals
and practices of this study's protocol.
Briefly, the steps involved were:
\begin{enumerate}
\item Introduction and Opening Questionnaire
\item Opening Interview
\item Setting Up/Recruitment
\item Android Study (Three weeks, with weekly Check-Ins)
\item Closing Interview
\end{enumerate}

  For a complete collection of the documents involved,
  the interview scripts, emails, questionnaires, and other exchanges,
  please see Appendix A.

\section{Introduction and Opening Questionnaire}
  For all participants,
  I first met with them to describe to them the study goals,
  my expectations from them, and to confirm their interest.
  If that all proceeded well, I walked them through the relevant paperwork,
  and administered an opening questionnaire,
  which included a few lines for basic information,
  a few questions about their habits,
  and a perceived stress scale.
  
\section{Opening Interview}
  \label{sec:opening_int}
  The opening interview, scheduled individually,
  aimed to answer a few high level research questions:
  \begin{enumerate}
  \item What are some common social support needs of individuals
  dealing with illness, bereavement, or other crises?
  \item In what ways do their current social interactions satisfy
  or fail to satisfy those needs?
  \item How is technology involved?
  \end{enumerate}

  The script of the interview was designed to answer those questions,
  and came in 5 parts.

  \begin{enumerate}
  \item \textbf{Background} -
    Understand their social support network,
    friends, family, and other mentors and the frequency of contact.
  \item \textbf{Support Needs and Topics} -
    This involved understanding the participant's stressors and how their
    support network is involved in helping to manage them.
    The last segment sought to understand how much the interviewee
    and the members of the interviewee's network valued
    mundane awareness.
  \item \textbf{Meeting Needs} -
    In this, I sought to understand to what degree,
    and in what ways, are their needs met or not met.
    I was interested in the useful tools and constricting barriers
    that were involved,
    and their specific methodology of seeing out support.
    This segment often involved interviewees explaining how they used technology
    and often segued into stories.
  \item \textbf{Technology} -
    This was to wrap up any further questions on existing technology use,
    particularly for participants that tended to focus on the human aspect,
    and less on the logistics of contact in earlier segments.
  \item \textbf{Motivation and Expectation} - 
    Wrapping up the interview, I asked about the things expected,
    coming into help me with the research,
    which helped me transition towards asking them if they had questions
    or concerns.
  \end{enumerate}

  In practice, the opening interview was conducted in spaces
  that the participants chose, to maximize their level of comfort.
  The interviewees were encouraged to take any tangents they felt
  were relevent or interesting, and thus,
  in addition to learning about their more quantifiable, comparable practices,
  it also become a good opportunity to hear their individual stories.

\section{Android (+ Set Up)}
  After the interview, the participants were sent specific instructions
  for setting up the Android portion of the study.
  They had to invite at least two of their friends/family/other supporters
  to interact on the application.
  Once that was done, I distributed the InMind application file as
  an installable APK over email, which was then forwarded to their people.

  Field testing InMind was complicated by the fact that it involved
  significant self-motivated participation from the participants.
  Because this step and the previous step of scheduling interviews
  involved significant delays,
  we chose to accept participants on a rolling basis,
  and allowed the field testing period to begin at a pace allowable
  by the individual participant.
  The target was three solid weeks of application usage,
  and these three weeks started as early as mid March
  and as late as early April.

  Once the application was distributed,
  I marked the date down as that particpant's start date.
  They would keep the application installed and in use for up to 3 weeks,
  at which point they would be asked to uninstall it.

\section{Questionnaires and other Feedback}
  At the end of every week of the study,
  I sent out a weekly questionnaire to the participants.
  The questionnaire had two parts:
  \begin{enumerate}
  \item \textbf{Written Feedback} -
  This short questionnaire asked for
  general usage feedback and thoughts.

  \item \textbf{Affective Benefits and Costs Questionnaire} -
  designed by Ijsselsteijn et al.\ in 2009,
  it is a validated questionnaire that
  addresses factors related to the costs and benefits of communication.
  Some factors in the costs category included obligations,
  expectations, threates to privacy, and process effort.
  The benefits included personal effort,
  thinking about each other,
  situational awareness,
  sharing experiences,
  staying in touch,
  recognition,
  and group attraction.
  \end{enumerate}

\section{Closing Interview}
  The closing interview followed the same pattern as the opening interview.
  The research questions to answer were:
  \begin{enumerate}
  \item How did InMind fit into their existing support network and methods?
  \item What were the perceived costs and benefits of integrating InMind
  into daily life?
  \end{enumerate}

  I chose to use the same five sections,
  but asked questions more directly relevant to their experience with InMind.
  \begin{enumerate}
  \item \textbf{Background} - 
  General feedback on experience using InMind,
  potentially piggybacking off of information provided via questionnaire
  or other feedback.
  \item \textbf{Support Needs and Topics} - 
  Asking about changes in the participants' lives over the course of the study,
  as well as which topics were useful to discuss over InMind.
  \item \textbf{Meeting Needs} - 
  Assess InMind's effectiveness as a tool to get support.
  \item \textbf{Technology} - 
  Assess if InMind influenced other communication methods.
  \item \textbf{Motivation and Expectations} - 
  General feedback on the experience during the study.
  \end{enumerate}

\section{Data Collection}
  \subsection{Database}
    Most of the data that is used is directly pulled from
    the InMind server's database.
    The database logs all objects and datatypes,
    along with creation and modification times.

  \subsection{Google Analytics}
    Some of the app timings do not involve contacts with the server.
    I used Google Analytics logging tools to record the time spent
    on each of the views.
