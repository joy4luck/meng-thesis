\chapter{User Study}
To evaluate the relevance of InMind as well as better understand
the social support needs of individuals,
I conducted a user study which was the bulk of the work for this thesis.

In this section, I will describe the goals and protocols of the user study,
and then briefly describe some qualities of the users that participated
in the study.

\section{Study Protocol}
  For this section, I will broadly outline the specific goals
  and practices of this study's protocol.
  For a complete collection of the documents involved,
  the interview scripts, emails, questionnaires, and other exchanges,
  please see Appendix A for a complete collection of materials.

  \subsection{Introduction and Opening Questionnaire}
  
  \subsection{Opening Interview}
  The opening interview aimed to answer a few high level research questions:
  \begin{enumerate}
  \item What are some common social support needs of individuals
  dealing with illness, bereavement, or other crises?
  \item In what ways do their current social interactions satisfy
  or fail to satisfy those needs?
  \item How is technology involved?
  \end{enumerate}

  The script of the interview was designed to answer those questions,
  and came in 5 parts.

  \begin{enumerate}
  \item \textbf{Background} -
    Understand their social support network,
    friends, family, and other mentors and the frequency of contact.

  \item \textbf{Support Needs and Topics} -
    This involved understanding the participant's stressors and how their
    support network is involved in helping to manage them.
    The last segment sought to understand how much the interviewee
    and the members of the interviewee's network valued
    mundane awareness.

  \item \textbf{Meeting Needs} -
    In this, I sought to understand to what degree,
    and in what ways, are their needs met.
    I was interested in the useful tools and constricting barriers
    that were involved,
    and their specific methodology of seeing out support.
    This segment often involved technology and often segued into stories.

  \item \textbf{Technology} -
    This was often to wrap up any further questions on existing technology use,
    particularly for participants that tended to focus on the human aspect,
    and less on the logistics of contact in earlier segments.

  \item \textbf{Motivation and Expectation} - 
    Wrapping up the interview, I asked about the things expected,
    coming into help me with the research,
    which helped me transition towards asking them if they had questions
    or concerns.
  \end{enumerate}

  In practice, the opening interview was conducted in spaces
  that the participants chose, to maximize their level of comfort.
  The interviewees were encourages to take any tangents they felt
  were relevent or interesting, and thus,
  in addition to learning about their more quantifiable, comparable practices,
  it also become a good opportunity to hear their individual stories.

  \subsection{Questionnaires}

  \subsection{Field Testing InMind}

  \subsection{Closing Interview}

\section{Users}

  \subsection{Recruitment}

  \subsection{Retention}

\section{Data Collection - TBD}
  \subsection{Database}
    Most of the data that is used is directly pulled from
    the InMind server's database.
    The database logs all objects and datatypes,
    along with creation and modification times.

  \subsection{Google Analytics}
    Some of the app timings do not involve contacts with the server.
    I used Google Analytics logging tools to record the time spent
    on each of the views.
