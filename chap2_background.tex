\chapter{Background}
\section{Related Research}
  \subsection{Transitions and Social Support}
    From the theme of bereavement, we generalized out to address the stress of transitions.
    Our focus is on the transitions related to bereavement and the diagnosis of chronic illness,
    but there is a broad range of transitions that one goes through,
    over the course of life.

    In general, transitions are a psychological process of reorientation, transformation, adaptation
    that often involve developing new identities and new ways of living
    \cite{kralik_06}
    In light of this definition, transitions are transformative processes,
    which are ultimately important for one's growth.
    However, as looking at bereavement and illness prepared us for,
    transitions have also been associated with increased levels of stress and decreased well being.
    \cite{mikal_13}
    Mikal et. al. in 2013, conducted a survey of different kinds of transitions,
    exploring them through a survey of many different articles addressing transitions and
    the shifts that individuals go through.

    In their survey, Mikal et al categorized transitions into the categories
    individual, family, community, and societal.
    They further discussed how specific types of transitions are either voluntary or not, anticipated or not,
    normative or disruptive, positive or negative, independent or interdependent.
    
    Social support is what people call on to deal with life's difficulties,
    whether it's friends, collegues, families, or anyone else, the people in our lives
    is what helps get us through transitions.
    
    However, during transitions is often when these support mechanisms can weaken or fail.
    \cite{mikal_13}

  \subsection{Positive Influences of Social Support}
    Social support is an important resource during times of transition and stress.
    One gets from friends and family support in the form of
    information, material things, and socio-emotional support,
    which are all important in helping one cope with difficulty.
    The impact of this support has manifested itself in a broad range of studies.
    Rains and Young found that supportive communication can help manage uncertainty and
    help one feel in control of one's life \cite{rains_09},
    and Shor et all even discovered that support from
    family is correlated with lower mortality. \cite{shor_13}

    We get agreement in the psychology of bereavement.
    Vachon and Stylianos studied the role of social support in bereavement back in 1988
    and discovered that an inadequate social network is associated with higher distress
    over the course of bereavement,
    and proposed that increasing support can decrease distress.

    However, it is also important to recognize that not all support is the same.
    Lehman et al studied in 1986 the ways that attempts at support that were helpful and unhelpful.
    \cite{lehman_86}
    When Vachon and Stylianos explored the "goodness of fit" between offered support and the actual needs of
    the bereaved, they discovered that it varies wildly, and is sensitive to variables such as impact of loss
    and presence of other stressors.
    \cite{vachon_88}
    
    As the complexity of personal networks suggests, the resources that people pull on
    are very specific,
    and their appropriateness is a complex subject.

  \subsection{Obstacles}
    % TODO: Flynn - Underestimating Compliance
    % TODO: Lee - tough don't ask for help.

  \subsection{A Role for Technology}
    % TODO: HJO - SNS and feelings of support.
    In light of these obstacles, there's a lot that technology can help with.
    Mikal et al suggest that computer mediated communication (CMC)
    and computer mediated social support (CMSS) can be quickly built up during or in the wake of
    a transition to help an individual cope.
    \cite{mikal_13}
    
    Interactions through the internet have shown to reduce feelings of depression, loneliness,
    and improve feelings of social support and self-esteem.
    \cite{shaw_02}




\section{Related HCI Work}
  \subsection{Chronic Disease Support}
  %TODO SKEELS - breast cancer help.
  %TODO Granholm - MATS, schizophrenia


  \subsection{Communication Technologies}
    When it comes to the practical aspect of designing for interpersonal
    communication, researchers in family interactions have significant experience.
    Designing technologies for a multi-person group is unusual because the usual
    standards of usability need to be applied to the unit as a whole, and not just
    the experience of a particular individual. \cite{neustaedter12}

    I am interested in technologies classified as "interpersonal awareness systems,"
    designed to be used by families and couples that are separated by distance.
    Systems like this help people maintain awareness of  of each other to connect
    and comfort each other. \cite{neustaedter06}
    In 2004, Markopoulos designed and tested an
    awareness system for families that shared pictures and snippets from a mobile
    device, demonstrating the value of mediating that connection through technology.
    \cite{markopoulos04}
    My theory is that the support of intimate family relations is a powerful
    resource that could
    have a very positive influence on the experience of stressed individuals.
