\chapter{Background}
\section{Related Research}
  \subsection{Problem Definition}
  The problem that we're addressing is broadly of stress and connectivitiy.
  From the theme of bereavement, we generalized out to address

  % TODO: MIKAL transitions and stress.
  % TODO: Shor - support from family lowers mortality

  \subsection{Obstacles}
  % TODO: Flynn - Underestimating Compliance
  % TODO: Lee - tough don't ask for help.

  \subsection{Positive Influences}
  % TODO: Vachon - social support and bereavement
  % TODO: Lehman - fguring out what is useful to bereaved.
  % TODO: HJO - SNS and feelings of support.

\section{Related HCI Work}
  \subsection{Chronic Disease Support}
  %TODO SKEELS - breast cancer help.
  %TODO Granholm - MATS, schizophrenia


  \subsection{Communication Technologies}
    When it comes to the practical aspect of designing for interpersonal
    communication, researchers in family interactions have significant experience.
    Designing technologies for a multi-person group is unusual because the usual
    standards of usability need to be applied to the unit as a whole, and not just
    the experience of a particular individual. \cite{neustaedter12}

    I am interested in technologies classified as "interpersonal awareness systems,"
    designed to be used by families and couples that are separated by distance.
    Systems like this help people maintain awareness of  of each other to connect
    and comfort each other. \cite{neustaedter06}
    In 2004, Markopoulos designed and tested an
    awareness system for families that shared pictures and snippets from a mobile
    device, demonstrating the value of mediating that connection through technology.
    \cite{markopoulos04}
    My theory is that the support of intimate family relations is a powerful
    resource that could
    have a very positive influence on the experience of stressed individuals.
