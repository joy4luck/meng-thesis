\chapter{Background}
  The two major goals that we outlined,
  sharing status and story, led us to two broader themes.
  \begin{enumerate}
  \item Interpersonal Awareness through Technology.
  \item Catalyzing Support During Crises
  \end{enumerate}

\section{Interpersonal Awareness through Technology}
  The sharing of statuses led us to the first theme:
  awareness through technology.

  Sharing statuses,
  in the form of availability or affective state,
  was really just one example of providing awareness that was valuable to loved ones.
  Technology is used in many different ways for people to maintain
  awareness of people with whom they have relationships,
  in the workplace or at home,
  and there are correspondingly many opportunities for design.

  \subsection{Awareness Systems}
    When it comes to the practical aspect of designing for interpersonal
    awareness, researchers in family interactions have significant experience.
    Designing technologies for a multi-person group is unusual because the usual
    standards of usability need to be applied to the unit as a whole, and not just
    the experience of a particular individual. \cite{neustaedter12}

    I am interested in technologies classified as "interpersonal awareness systems,"
    designed to be used by families and couples that are separated by distance.
    Systems like this help people maintain awareness of  of each other to connect
    and comfort each other. \cite{neustaedter06}
    In 2004, Markopoulos designed and tested an
    awareness system for families that shared pictures and snippets from a mobile
    device, demonstrating the value of mediating that connection through technology.
    \cite{markopoulos04}

  \subsection{Other Non-Verbal Communication}
    \\TODO

\section{Support During Crisis}
  The sharing of stories,
  in the context of using them for reflection and sharing serious thoughts,
  led us to a second theme:
  support during crises,
  specifically social support and helping.

  Sharing stories is one of many ways that
  individuals reach out during life crises.
  and only one of many ways they can be helped.
  Individuals dealing with other major life events or transitions
  reach out to loved ones with many variations of serious stories,
  and helpful responses can take many different forms.

  However, as the complexity of personal networks suggests, the resources that people pull on
  are very specific,
  and their appropriateness is a complex subject.

  It is also important to recognize that not all support is the same.
  Lehman et al studied in 1986 the ways that attempts at support that were helpful and unhelpful.
  \cite{lehman_86}
  When Vachon and Stylianos explored the "goodness of fit" between offered support and the actual needs of
  the bereaved, they discovered that it varies wildly, and is sensitive to variables such as impact of loss
  and presence of other stressors.
  \cite{vachon_88}
  
  Further, stressful transition periods can be accompanied by changes in the social structure
  that the support depends upon.
  The situation of illness or bereavement obviously includes, necessarily,
  the loss of an individual that would otherwise be a source of support,
  but many other transitions can also cause rearrangement of personal networks.
  Consequently, during transitions is often when these much needed
  support mechanisms can weaken or fail.
  \cite{mikal_13}

  Technology can help to mediate the appropriate type of resilient support
  that individuals may need.

  \subsection{Social Support}
    Social support is an important resource during times of transition and stress.
    Whether it's friends, collegues, families, or anyone else, the people in our lives
    are what help get us through transitions.
    \cite{mikal_13}
    One gets from friends and family support in the form of
    information, material things, and socio-emotional support,
    which are all important in helping one cope with difficulty.
    The impact of this support has manifested itself in a broad range of studies.
    Rains and Young found that supportive communication can help manage uncertainty and
    help one feel in control of one's life \cite{rains_09},
    and Shor et all even discovered that support from
    family is correlated with lower mortality. \cite{shor_13}

    The most active community developing software for social support is,
    not surprisingly, in the realm of support for chronic diseases.

    In 2010, Skeels et. al used a participatory design methodology to explore what is most useful
    to breast cancer patients.
    The support of intimate family relations is a powerful
    resource that could
    have a very positive influence on the experience of individuals going through
    tough transitions.
    With breast cancer survivors and their friends, Skeels et al. compiled a model for how
    suggestions and requests for help can be compiled and managed by patients \ref{fig:skeels_diagram},
    and collected all the many barriers to social support that are common for breast cancer patients.
    The prototype technology they implemented was a website, supported by Facebook Connect,
    that allowed patients to perform those request/help idea management tasks.
    
  \subsection{Support through Mobile Technology}
    Mobile technology is also entering into a realm of new intimacy,
    and is in a unique place for intervention.

    In 2011, Granholm et. al designed a mobile intervention,
    effectively a text messaging service, that helped patients diagnosed with
    schizophrenia.
    The text messaging service effectively texted the patients 4 times a day,
    asking questions or giving them a suggestion.
    The results showed significant improvement for medication adherence for those living alone,
    and improvements in socialization and a decrease in auditory hallucinations. \cite{granholm_12}

  \subsection{Existing Services}
    There are several online services that support patients and their supporters.
    Caring Bridge (found at http://www.caringbridge.org/),
    CarePages (found at https://www.carepages.com),
    and The Status (found at https://www.thestatus.com) are three such services.
