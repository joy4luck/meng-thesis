% $Log: abstract.tex,v $
%
%% The text of your abstract and nothing else (other than comments) goes here.
%% It will be single-spaced and the rest of the text that is supposed to go on
%% the abstract page will be generated by the abstractpage environment.  This
%% file should be \input (not \include 'd) from cover.tex.

Individuals under unusual stress, whether due to bereavement, illness,
serious conflict, or other transitions can benefit from social support.
Support is a valuable resource,
but there are many psychological and technical obstacles
that can get between an individual
and those he gets support from.

To compliment and improve upon existing communication strategies,
I designed InMind, an Android application
that mediates life-issue tracking
between close relations.
InMind helps individuals communicate
the status of important topics in their lives and 
express their thoughts related to the topics,
adding to awareness of these issues.

In my thesis work, I conducted interviews and field tested InMind
to explore the social support needs of
these individuals. I consider how needs are met or not met,
how technology is involved, and what can help.

We deployed InMind for 3 weeks, with 7 groups of size 3 to 4 each.
Only one group had full participation, 
while the others were dominated by 1 or 2 participants.
By the end, over 1000 messages were sent by participants.
A majority of the users brought topics from within InMind to other media,
such as emails or phone calls.

Interview results from 15 participants showed strong
asymmetry in awareness and different preferences for degree of sharing
and support from loved ones.
Participants were split between those who were sensitive to the privacy
of their status and a few who would prefer that everything be public.
Since there were often only a few active participants within a group,
the active participants found value in using InMind for self-reflection.
Conducting studies that study strong-tie relationships
have to overcome the difficulty of getting participation from each member
of the relationship.
