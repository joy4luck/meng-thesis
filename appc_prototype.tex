\chapter{InMourning Prototype}

\section{Background: Bereavement Technologies}
  For inspiration and guidance for this initial prototype,
  I looked into existing bereavement technologies.
  Research into bereavement from the perspective of HCI is young, but the
  existing literature reveals some of the core problems that technology can
  address. Among the earliest research is that of Massimi and Baecker in 2010
  \cite{mm10},
  a needs-analysis of all the ways that technology intersects with the lives of
  the bereaved.
  Part of their study was an observation of how,
  in the potentially-dispersed setting of the modern world, 
  technology is already used to help bring people together during mourning.
  Massimi and Baecker proposed a wide set of design opportunities and challenges.
  They were primarily interested in exploring technological artifacts,
  both inherited from the deceased and those created by the bereaved to
  remember and grieve, but they also mention the possibility of technology
  helping to provide social support.

  % Homescreen Figure
  \begin{figure}
  \caption[Screenshot from Massimi et al.\ Besupp]
  {\textbf{Besupp Screenshot} --
  Besupp offered circles for people with like losses to share thoughts
  and feelings with each other.
  }
  \centering
  \includegraphics[width=0.65\textwidth]{besupp.png}
  \label{fig:besupp}
  \end{figure}

  There have been few technologies designed explicitly for the grieving. One
  example is Besupp, an online platform for maintaining a support group between
  people who have experienced similar losses, designed and tested by Massimi and
  co. in 2013. \cite{mm13}
  A screenshot from the study is shown in Figure \ref{fig:besupp}.
  Massimi and co. observed that technologies can be helpful as
  one of many ways to help with coping, as the needs of the bereaved are best
  addressed by a variety of interactions.
  This further study demonstrated that
  there is value in support systems that are maintained over long periods and
  designed for to low and intermittent usage.

  Other research has generally been observational, for instance Brubaker's study
  in 2011 of usage of existing technologies. \cite{brubaker11}

\section{Original Application Design}
On the home screen, the user sees two tabs,
which direct them to each of the two functions of the application.
(Figure \ref{fig:homescreen})

% Homescreen Figure
\begin{figure}
\caption[InMourning Home Screen]{\textbf{Application Homescreen} --
The homescreen has two tabs, one for status updates and one for stories.
The status page shown currently shows the user's own status,
but can be expanded to have many users' current status.
The story page shown has a list of stories written by the user. There are photographs
when they are available, otherwise a small candle icon stands in.
}
\centering
\includegraphics[width=0.75\textwidth]{homescreen.png}
\label{fig:homescreen}
\end{figure}

The first function is to aid in communication management.
The application supports efficient input of the user's current availability
and preferred method of
contact, if any, and broadcasts it to any subscribers.
The current design changes an avatar that represents availability and mood.
(Figure \ref{fig:update})

% Update Figure
\begin{figure}
\caption[InMourning Status Update]{\textbf{Status Update Window} --
The status update window lets the user toggle between three moods
and three availability states.
Three representative combinations are shown, but any is possible.
}
\centering
\includegraphics[width=0.90\textwidth]{update.png}
\label{fig:update}
\end{figure}

Availability is indicated pictorially by a door that can be opened and closed;
mood is indicated pictorially by a smiley face whose expression changes.
The preferred method of contact will be included in the message that
accompanies a status.

The second function is to share stories and affect.
The application allows the user to input pictures and/or text and
choose who the recipients are.
A technology to support these interactions would consist of a mobile
application and a supporting server. 

In the initial prototype, the focus is on giving users space to write out
their story.
The mobile interface is largely an open space for writing,
as Figure \ref{fig:compose} shows.

% Compose Figure
\begin{figure}
\caption[InMourning Story Compose]{\textbf{Story Compose Window} --
The story sharing window lets the user title their story,
write their story,
and optionally add a photo.}
\centering
\includegraphics[width=0.95\textwidth]{compose.png}
\label{fig:compose}
\end{figure}

\section{Maturation into InMind}
  However, we soon ran into a problem.
  If we want to use technology to improve the relationships between friends and family during the
  process of bereavement, we have to introduce technology well before the crisis happens.
  This is for primarily two reasons.
  \begin{enumerate}
  \item Death is often a stunning, overwhelming catastrophe,
  and bereaved individuals cannot be burdened with learning how to use a new technology.
  Many recently bereaved individuals already find it difficult to do things they normally can. \cite{parkes13}
  \item For the deaths that are more forseeable, such as due to illness,
  there is often a long period of handling the dire situation,
  in which many of the same needs exist.
  \end{enumerate}

\clearpage
\newpage
