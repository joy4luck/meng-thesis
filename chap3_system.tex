\chapter{System Overview}
  The system designed to address a subset of these concerns is essentially a mobile application.
  I designed and implemented,
  from the ground up, an application called InMind.
  InMind would enable the sort of slow,
  targeted communication that had been useful to bereaved individuals in forums,
  but connect individuals with people that they choose.

  Broadly, the goal of the application was to enable users to create spaces
  with themed topics, that had a status of some nature.
  This status could represent the user's well being, the well being of someone else
  that the user cares about, the progress of a project, the energy level of a group,
  the level of anxiety the user is suffering from, etc.
  Since mobile applications are always accessible,
  these themed spaces could be updated at any point
  during the day when the status of that topic changes.

  Finally, since individuals have shown to be sensitive
  to which particular people they share
  sensitive topics with \cite{},
  the users can choose when they begin the topic
  which of their people they want to share with.

  To summarize, once shared, the topics would be defined by several characteristics:

  \begin{enumerate}
  \item \textbf{Title} - Answers the question of \textbf{What}. The user can choose what to call the topic.
  \item \textbf{Share With} - Answers the question of \textbf{Who}.The topic is like a static chatroom,
    since the people who are listening in are always the same.
  \item \textbf{Status} - Answers the question of \textbf{How}. The "status icon" of the topic sets the mood
    of the topic. Unlike for chatrooms,
    this status allows the space to reflect the state
    of the user who started the topic.
  \end{enumerate}

  Within those spaces, we wanted to encourage communication.
  Here, although I was inspired by the nature of online forums,
  where many people reach out for support,
  our goal was more influenced by the nature of mobile communications.
  On a mobile device, users expect to use abbreviated conversation.
  Although the thoughts may be complex, the messages composed on a mobile device
  tend to be very short.

  \section{Application End}

    \subsection{Platform Choice}
    The application was written for Android, targeting API 18,
    but supports down to API 13.

    I chose to develop in Android primarily because the Affective Computing Group
    has a recent tradition for doing so.
    My primary reason for choosing Google Android over Apple iOS include
    the prevalence of Android devices.
    Android has been dominating the mobile market since 2012,
    and if InMind were to be released to the public,
    an Android application is more likely to reach out to a broader audience.

    The InMind application supports API 13 through 18, 
    corresponding, respectively, to Android 3.2 Honeycomb and 4.3 Jellybean.

    Android 4.3, Jellybean, was released the year ago, but is by far
    the most prevalent in the MIT target population due to the devices
    that supported it sold over the course of the year.
    During development, Android 4.4, KitKat is being pushed out to devices,
    but yet isn't available to a significant degree.

    At the opposite end, API 13, was released with Android 3.2 HoneyComb
    in 2011, 3 years prior to this date of writing.
    Mobile devices have a life span of about two years,
    and thus it is unlikely for Android users to be
    running versions of Android older than API 13.

    \subsection{Hardware}
    InMind has to integrate into a user's existing communication methods,
    as that is the environment actually encountered in the lives of people.
    Thus, emulating more typical applications,
    InMind is designed to be installed onto the user's existing mobile device.

    As previously suggested by the range of compatible Android versions,
    InMind is designed to be run on an assortment of devices.
    
    In addition to version issues, Android devices vary significantly in form factor.
    The two most popular are cell phone-like devices, which have roughly a 4" screen,
    and small tablets, which have roughly a 7" screen.

    I developed primarily on two devices, a Samsung S4, with a 4" screen,
    and a 2012 Nexus 7, with a 7" screen.
    Design objectives for all layouts was to be attractive on the cell-phone-sized screens,
    and reasonably functional on larger tablet screens.

    \subsection{Security}


  \section{Server}
    \subsection{Design}
      The server for InMind was written using Node.js.
      The choice of Node.js offered many benefits, including asynchronous responses,
      lightweight maintenance, and easy integration with MongoDB for data retention.

      MongoDB gave us flexibility for designing how we wanted data to be stored,
      and we took advantage of MongoDB\'s simple backup mechanism for data integrity.

      The data saved on the server included records of all the app interactions.
      For each of the actions performed by the users,
      we logged the timestamp of the operation as well as the recipients.

      To keep user data secure, the server was hosted on an MIT Media Lab server.
      The machine... TBD

    \subsection{Security}
      During the study, while the server was running, it was password protected,
      as was the database in which the data was stored.

      For added protection for our users, all messages were encrypted.
      This encryption was on several levels, primarily using the AES standard.
      TBD

      \begin{enumerate}
      \item To protect user identities, signup for the lead participant was performed manually.
      All "lead" participants had to be entered into the database manually,
      and later participants sign up anonymously with a user id and password.
      Data on all users are deidentified;
      the database remembers only an alphanumeric user id.
      \item Each group shares an initialization vector.
      \item Each topic has a salt and an autogenerated passphrase.
      \end{enumerate}



  \section{Design Challenges}
    \subsection{Lightening the burden of communication}

    \subsection{Technology as an excuse}


