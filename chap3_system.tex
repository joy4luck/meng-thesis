\chapter{System Overview}

  \section{Application}
    \subsection{Hardware}

    \subsection{Security}


  \section{Server}
    \subsection{Design}
      The server for InMind was written using Node.js.
      The choice of Node.js offered many benefits, including asynchronous responses,
      lightweight maintenance, and easy integration with MongoDB for data retention.

      MongoDB gave us flexibility for designing how we wanted data to be stored,
      and we took advantage of MongoDB\'s simple backup mechanism for data integrity.

      The data saved on the server included records of all the app interactions.
      For each of the actions performed by the users,
      we logged the timestamp of the operation as well as the recipients.

      To keep user data secure, the server was hosted on an MIT Media Lab server.
      The machine... TBD

    \subsection{Security}
      During the study, while the server was running, it was password protected,
      as was the database in which the data was stored.

      For added protection for our users, all messages were encrypted.
      This encryption was on several levels, primarily using the AES standard.
      TBD

      \begin{enumerate}
      \item To protect user identities, signup for the lead participant was performed manually.
      All "lead" participants had to be entered into the database manually,
      and later participants sign up anonymously with a user id and password.
      Data on all users are deidentified;
      the database remembers only an alphanumeric user id.
      \item Each group shares an initialization vector.
      \item Each topic has a salt and an autogenerated passphrase.
      \end{enumerate}



  \section{Design Challenges}
    \subsection{
    \subsection{Lightening the burden of communication}

    \subsection{Technology as an excuse}


